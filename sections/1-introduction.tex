%% INTRODUCTION SECTION %%

\section{Introduction}

Conversational agents (CAs), such as Amazon Alexa or Google Assistant, are designed to interact with humans using natural language through voice or text modalities. Since these agents communicate in natural language, \textit{conversation architecture}---what and how a CA says---affects users' perceptions of agents and, in turn, the effectiveness of their interactions with users \cite{knijnenburg2016inferring, seeger2021chatbots}, i.e. supporting users in achieving their goals. For instance, previous research suggests that there is a relationship between CAs' conversation architecture and how users perceive these agents %.
%While various studies investigated how some elements of conversation architecture affect users' perceptions of agents~
(e.g.,~\cite{clark2019state, feine2019taxonomy, finch2020towards}).  However, there is currently a lack of systematic understanding of these relationships, i.e. what elements of conversation architecture are impactful for users' perceptions and how. Specifically, while various studies investigated how some elements of conversation architecture affect aspects of users' perceptions or discussed the speech specifics of agents as part of a broader analysis of the state of research \cite{clark2019state,feine2019taxonomy}, there is no comprehensive analysis of %research specifically for 
conversation architecture elements that are relevant to user perceptions. Yet, this systematic understanding is essential for guiding the design of conversation architecture, which would allow us to both predict and develop users' appropriate perceptions of agents.

%there is a lack of systematic understanding of these relationships.
%paucity of knowledge on the overall coverage and effects of these relationships explored in literature. 
%One reason for this disparity is, arguably, the lack of clear and consistent definitions of both the aspects of users' perceptions of agent and conversation architectural elements. Although some authors discuss the speech specifics of agents as part of a broader analysis on the state of research \cite{clark2019state,feine2019taxonomy}, there is no comprehensive analysis of %research specifically for 
%conversation architecture elements that are relevant to user perceptions. Similarly for user perceptions, while there is some research exploring the perceptions of agents as part of understanding the user experiences \cite{clark2019state, finch2020towards} or social cues \cite{feine2019taxonomy} of CAs, there is no exhaustive research specifically for perceptions of agents that are affected by conversation architecture elements. 

%These research gaps related to the paucity of knowledge on synthesizing the effect of speech specifics on perceptions of CAs as well as the lack of systematic understanding on the elements of conversation architecture and perceptions of agents 
To address this gap, %in this paper, we
%we aim to understand how to design conversation architecture in order to orchestrate perceptions of conversational agents.
%First, we 
we explore the following research questions in this paper: \textbf{(RQ1) what aspects of users' perceptions of agents have been explored in relation to the effects of conversation architecture elements?}, \textbf{(RQ2) what elements of conversation architecture are relevant to users' perceptions of agents?}, and \textbf{(RQ3) what is known about the specifics of these relationships between conversation architecture elements and the perceptions of CAs?}
%While there are various elements of conversation architecture used in a CA's utterances, we don't have a systematic understanding of them. This leads to our second research question: \textbf{(RQ2) what elements of conversation architecture are relevant to users' perceptions of agents?} Lastly, we want to understand the effect of conversation architecture on perceptions of agent, specifically: \textbf{(RQ3) what are the relationships between conversation architecture elements and the perceptions of CAs?}
To answer these questions, we performed a comprehensive review of the existing literature published from 2010 to 2022 and synthesized the effect of conversation architecture elements on the perceptions of CAs based on the final dataset of 57 relevant papers (\autoref{fig:heatmap_identified}). Based on this review, we grouped the perceptions of agents that were shown to be affected by conversation architecture into four categories (interaction, ability, sociability and humanness) and eleven aspects (\autoref{tbl:perceptions}), and grouped the speech variations related to perceptions of agents into four categories (dialog strategy, content affectiveness, content style and speech format) and eleven elements (\autoref{tbl:conversation_architecture}). 
We found that perceptions of interaction with CAs (usability, engagement, satisfaction) have been most explored in relation to speech variations, while the perceptions of CAs' ability (intelligence, competence, credibility) have been explored the least.
%the least number of explored and identified relationships. 
Our results also show that the conversation architecture elements of response delay and agent-initiated dialog related to perceptions of CAs are under-explored in literature. 
Thus, our work contributes to the CUI community by presenting a framework that synthesizes the findings on the effect of conversation architecture on perceptions of conversational agents while also identifying and clustering perception aspects and conversation architecture elements. %, providing clear and consistent terms that can be used in future research. Together, these contributions move us one step closer to designing CAs that effectively interact with users. 

In the remainder of the paper, we first outline related background work, then describe the literature review process and the framework synthesis method, %, the characteristics of the papers in our reviewed corpus. Following that, we 
followed by a detailed description of the taxonomies of the aspects of the perception of agents and conversation architecture elements, as well as the synthesized framework for their relationships. We conclude by discussing the implications of this work, including 
the associated research challenges and opportunities, as well as ethical considerations for designing the conversational architecture to affect users' perceptions of conversational agents.