%% INTRODUCTION SECTION %%

\section{Introduction}

Conversational agents (CAs), such as Amazon Alexa or Google Assistant, are designed to interact with humans using natural language through voice or text modalities. Since these agents communicate in natural language, \textit{conversation architecture}---what and how a CA says---affects users' perceptions of agents and, in turn, the effectiveness of their interactions in achieving users' goals \cite{knijnenburg2016inferring, seeger2021chatbots}. According to previous research, there is a correlation between CAs' conversation architecture and users' perception of these agents.

While various studies investigated how different speech elements affect agents' perceptions, there is a paucity of knowledge on the overall coverage and effects of these relationships explored in literature. One reason for this disparity is the lack of clear and consistent definitions of the aspects of users' perceptions of agent and conversation architectural elements. Some papers have discussed the speech specifics of agents as part of  a broader analysis on the state of research \cite{clark2019state} or social cues \cite{feine2019taxonomy}, but there is no comprehensive research specifically for conversation architecture elements that are relevant to user perceptions. Similarly for user perceptions, while there some research exploring the perceptions of agents as part of understanding the user experiences \cite{clark2019state, finch2020towards} or social cues \cite{feine2019taxonomy} of CAs, there is no exhaustive research specifically for perceptions of agents that are affected by conversation architecture elements. 

%While various studies investigated how different conversation architecture elements affect agents' perceptions, there is a paucity of knowledge on the overall coverage and effects of these relationships explored in literature. For conversation architecture elements, the use of affective language is commonly explored, with studies finding that conversational agents are perceived as more socially present and emotionally intelligent if they use sentiment-adaptive response based on user's utterances~\cite{diederich2019emulating, yang2017perceived}. Another commonly explored element is the use of prosody in voice-based agents, such as the use of express prosody contributing to higher perceived enjoyment and intimacy with the agent \cite{kim2020can}. However, there is a lack of research to synthesize these findings into an overall framework. One of the possible issues is the lack of clear and consistent definitions of the aspects of users' perceptions of agent and conversation architectural elements.

%Currently, there is no systematic understanding on the elements of conversation architecture that are relevant to the perceptions of agents. Some papers have started to examine the speech specifics explored in literature, such as \citet{clark2019state} capturing different elements of system speech production for voice-based agents and categorizing them based on content, style, and other topics. There are also some investigations of conversation architecture elements within the broader field of studying cues that trigger a social reaction of the user towards the CA. Feine et al's \cite{feine2019taxonomy} paper laid the foundation to create a consistent taxonomy for these social cues. However, there is no comprehensive research specifically for conversation architecture elements that are relevant to user perceptions. 

%Related to user perceptions, there is also no systematic understanding on the perceptions of agents affected by speech variations. Some research discussed the perceptions of agents as part of the broader scope of user experiences of CAs. These publications have gathered UX dimensions used to evaluate conversational agents, which include the perceptions of agents (e.g. empathy, humanness) alongside other assessments (e.g. physiological data) \cite{clark2019state, finch2020towards}. Some even categorized the metrics based on the type of method used, including technical measures like system error rate and perception measures like perceived social presence \cite{zheng2022ux}. However, these studies are assessing the general field of UX measurements and not specifically on the perceptions of agents, and do not analyze the effects of conversation architecture on these perceptions.

These research gaps related to the paucity of knowledge on synthesizing the effect of speech specifics on perceptions of CAs as well as the lack of systematic understanding on the elements of conversation architecture and perceptions of agents motivated us to the goal of how to design conversation architecture in order to orchestrate perceptions of conversational agents. First, we want to understand \textbf{(RQ1) what perception aspects of agents are being explored in relation to the effects of conversation architecture elements}. While there are various architecture elements of conversation architecture used in a CA's utterances, we don't have a holistic understanding of them. This leads to our second research question: \textbf{(RQ2) what elements of conversation architecture are relevant to the perceptions of agents?} Lastly, we want to understand the effect of conversation architecture on perceptions of agent, specifically: \textbf{(RQ3) what are the relationships between conversation architecture elements and the perceptions of CAs?}

To answer these questions, we performed a comprehensive review on existing literature from 2010 to 2022 and synthesized the effect of conversation architecture elements on the perceptions of CAs based on 57 relevant papers (\autoref{fig:heatmap_identified}). We also grouped the perceptions of agents affected by conversation architecture into four categories and eleven aspects (\autoref{tbl:perceptions}), and grouped the speech variations related to perceptions of agents into four categories and eleven elements (\autoref{tbl:conversation_architecture}). 
We found that perceptions of interaction with CAs (usability, engagement, satisfaction) have the most number of explored and identified relationships with speech variations, while the perceptions of CAs' ability (intelligence, competence, credibility) have the least number of explored and identified relationships. Also, the conversation architecture elements of response delay and agent initiated dialog related perceptions of CAs are under-explored in literature.
%Lastly, we coded the explored connections between conversation architecture and perceptions and synthesized their relationships into a framework. Through our analysis, we also found that there are inconsistencies in the assessment of perceptions across studies, as well as influencing factors other than conversation architecture that impact the perceptions of agents. 
This work contributes to the HCI community by presenting a framework that demonstrates the effect of conversation architecture on perceptions of conversational agents, as well as taxonomies for perception aspects and conversation architecture elements providing clear and consistent terms that can be used in future research. Together, these contributions move us one step closer to designing CAs that effectively interact with users. 
%Also, our analysis extends existing reviews on UX research of CAs \cite{clark2019state}\cite{rapp2021human}\cite{zheng2022ux} by providing a synthesized framework outlining the relationships between conversation architecture and perceptions of agents. With the help of this framework, we are taking the first step to understand how to design conversational agents to solicit specific social reactions from users.

In the remainder of the paper, we first outline related background work, our literature review process, as well as the characteristics of the papers in our reviewed corpus. Following that, we describe in detail the taxonomy for the perception of agents, the taxonomy for conversation architecture element, as well as the framework on the relationships between perceptions of agents and conversation architecture. Finally, we discuss research challenges and opportunities as well as ethical considerations for designing conversational architecture to orchestrate perceptions of conversational agents.