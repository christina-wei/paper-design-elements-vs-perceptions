%% RELATED WORK SECTION %%

\section{Related Work}
%Communication with conversational agents 
Similar to human-human communication, human communication with conversational agents relies on natural languages for interaction. However, while it is tempting to leverage components from human-human conversation in designing interactions with conversational agents, studies have demonstrated that there are differences between human-human communications and human-agent communications \cite{clark2019makes, hill2015real}. Currently, CAs are considered as a user-controlled tool rather than a social companion, with a focus on the utilitarian aspects of the conversation \cite{clark2019makes}\cmt{[1]}. When conversing with agents, users tend to use shorter messages with limited vocabulary, adapting to the style of the agent \cite{hill2015real}. Also, some studies found that small talk and humour are considered unnecessary or even inauthentic in conversations with CAs, while these elements are important in scaffolding communications with human partners \cite{clark2019makes, doyle2019mapping}\cmt{[1],[2]}. There are also machine-like traits that users preferred in a CA, such as the ability to interact through multimodal media, and a machine's perceived ability to be objective and non-judgemental \cite{doyle2019mapping, kim2022understanding}\cmt{[2],[3]}. Given human-agent communications are different from human-human communications, it is important to look specifically at the design of user experiences for conversational agents.

There are a number of studies investigating factors affecting user experiences for conversational agents, which includes exploring the relationship between conversation architecture elements and agents' perceptions. For instance, the use of affective language is commonly explored, with studies finding that conversational agents are perceived as more socially present and emotionally intelligent if they use sentiment-adaptive responses based on user's utterances \cite{diederich2019emulating, yang2017perceived}. Another commonly explored element is the use of prosody in voice-based agents, such as the use of express prosody contributing to higher perceived enjoyment and intimacy with the agent \cite{kim2020can}. However, there is a paucity of research to synthesize these findings on user experiences with agents across literature. 

In recent years, several papers have synthesized the current state of research for conversational agents for specific agent types (e.g. text-based, voice-based, polyadic) highlighting major trends, topics, methods and evaluating metrics \cite{clark2019state, rapp2021human, zheng2022ux}\cmt{[33][5]}. Other papers looked at specific dimensions of user experiences or domains of usage, such as \citet{van2020human} examining the effects of agents' human-like communicative behaviours on their relationships with users, and \citet{kocaballi2022design} exploring the challenges and opportunities of CAs in healthcare. There are also publications aimed to unify concepts used in the research for CA user experiences, for instance \citet{feine2019taxonomy} creating a taxonomy of social cues, and \citet{finch2020towards} analyzing evaluation protocols for dialogue systems. To the best of our knowledge, there is currently no synthesis on the speech specifics of CAs, the perceptions of agents, or the relationship between them.

Conversation architecture has been demonstrated to play an important role in users' perceptions of agents \cite{knijnenburg2016inferring, moussawi2021perceptions, seeger2021chatbots}\cmt{[35]}. Yet, little is known about the overall landscape on the how the specifics of speech variations affect these perceptions. Given the paucity of research in this area, this paper aims to synthesize the identified relationships between conversation architecture elements and users' perceptions of agents.