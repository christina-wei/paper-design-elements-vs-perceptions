%% CONCLUSION SECTION %%

\section{Conclusion}

In this paper, we share our findings of a systematic review of existing literature published in the ACM Digital Library on the effect of conversation architecture elements on the perceptions of CAs. Through the synthesis of 57 papers in our corpus, we found that current literature has explored users' perceptions of agents related to interaction, ability, sociability and humanness in relation to the effects of conversation architecture (RQ1). Also, we observed that conversation architecture elements related to dialog strategy, content affectiveness, content style, and speech format are relevant to users' perceptions of agents (RQ2). Based on our in-depth analysis, we present a framework outlining the identified relationships between elements of agents' conversation architecture and aspects of users' perception (RQ3). Our investigation also revealed the need for consistent protocols in evaluating perceptions of agents, as measurements are inconsistent across studies. Also, more research is needed to investigate the under-explored areas in the framework, the relationship across perception aspects, the influence of contextual factors, and the effect of composite conversation architecture elements on users' perceptions. While our research contribute to the design of conversation architecture to orchestrate specific perceptions of agents, we urge designers to incorporate ethical perspectives into their design considerations, including potential gender stereotypes, the use of persuasive techniques to influence users, and privacy issues related to users disclosing sensitive information to agents.